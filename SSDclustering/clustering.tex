\newpage
\section{Clustering}
There are two distinct set of approaches to doing clustering without training:
\begin{enumerate}
\item cluster the feature vectors of a pre-trained model
\item cluster the input images using:
  \begin{enumerate}
   \item "2nd-order" region statistics: e.g. region covariance \ref{regcov} and Sigma-set \ref{sigmaset}
   \item some form of non-Local Means (NLM)
\end{enumerate}
\end{enumerate}

There the algorithm to perform the clustering also deserve some investigation.
K-means is certain a starting point but it is very sensitive to initialization.
Next, the question of producing a quasi balanced cluster should also be considered.

If we are using a NLM approach, naive approach will make $k$ passes for $k$ clusters
while testing for nearest center. At least we need a fast lookup structure so 
that the nearest center can be retrieved in $O(logn)$ or $O(1)$ time.
Or use the very nice SigmaSet \ref{sigmaset} or \cite{Kwatra2010}

\subsection{Region Covariance}\label{regcov}
Region covariance can be computed very efficiently using 'integral images/sumarea table" \cite{Porikli2006, Tuzel2006}. Let $F(x,y) = \phi(I,x,y)$ be the $W\times H\times d$ dimensional feature image extracted from $I$, where $\phi$ can be any mapping
such as intensity, color, gradients, filter response etc.

The region covariance $\mathbf{C}_R$ is a $d\times d$ matrix of the feature points:
\begin{equation}
 \mathbf{C}_R = \frac{1}{n-1}\sum\limits^n_{k=1}(\mathbf{z}_k - \mu) (\mathbf{z}_k - \mu)^T
\end{equation}
where $\mu$ is the mean. $\mathbf{C}_R$ is $d\times d$ instead of $n\times d$
if we are using the raw features. Also, RC does not have any information regarding
the ordering and the number of points. This implies a certain scale and rotation
invariance.

Covariance matrices do not lie on Euclidean space, a distance metric involving generalized eigenvalues follows from the Lie group structure of positive definite matrices:
\begin{equation}
 \rho(\mathbf{C}_1, \mathbf{C}_2) = \sqrt{\sum\limits^n_{k=1} ln^2 \lambda_i(\mathbf{C}_1, \mathbf{C}_2)}
\end{equation}
where $\lambda_i(\mathbf{C}_1, \mathbf{C}_2)$ are the generalized eigenvalues
of $\mathbf{C}_1$ and $\mathbf{C}_2$ computed from 
$\lambda_i\mathbf{C}_1\mathbf{x}_i - \lambda_i\mathbf{C}_2\mathbf{x}_i$
and $\mathbf{x}_i \ne 0$ are the generalized eigenvectors.

\href{https://spie.org/news/0368-using-covariance-improves-computer-detection-and-tracking-of-humans?SSO=1}{Using covariance improves computer detection and tracking of humans}


\subsubsection{Sigma set}\label{sigmaset}
\cite{Chang2009} proposes a novel 2nd-order statistics based region descriptor, named "Sigma Set", in the form of a small set of vectors, which can be uniquely constructed through Cholesky decomposition on the covariance matrix.
This is basically an optimized form of region covariance using sparsity.

for any matrix $A$ that satisfies $C_R=AA^T$, the set of columns of $A$ has the same 2nd order statistics as $R$. More specifically, we can construct the Sigma Set descriptor for region $R$ through Cholesky decomposition

\subsection{Kwatra2010}
"Fast Covariance Computation and
Dimensionality Reduction for Sub-Window Features in Images"

\subsubsection{Faulkner2015}
\cite{Faulkner2015} "A Study of the Region Covariance Descriptor: Impact of Feature Selection and Image Transformations"

\subsection{K-means}

\subsubsection{Balanced K-means}
\cite{Malinen2014} 
in k-means assignment phase, the algorithm solves the assignment problem by Hungarian algorithm with time complexity $O(n^3)$.

\subsubsection{Balanced K-means \& min-cut}
\cite{Chang2014}

\subsection{Non-Local Means}
\subsubsection{Qian2013}
\cite{Qian2013} 
nonlocal similarity and spectral-spatial structure of hyperspectral imagery into sparse representation. Non-locality means the self-similarity of image, by which a whole image can be partitioned into some groups containing similar patches. The similar patches in each group are sparsely represented with a shared subset of atoms in a dictionary making true signal and noise more easily separated.

\subsubsection{Fu2017}
\cite{Fu2017}